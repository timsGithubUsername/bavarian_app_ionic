% !TEX root = latex_main.tex

\section{Anhang}
\subsection{Script der Vorstellung}
Die Bavarian App ist ein Lernprogramm um verschiedene Bayrische Dialekte ausgehend von unterschiedlichen Sprachen wie Deutsch, Arabisch oder Tschechisch zu lernen. Es gibt zwei Modi, einen um den ausgewählten Dialekt zu lernen und einen um sich diesem Dialekt in einem Quiz zu stellen. Ein Dialekt ist in verschiedene Kategorien aufgeteilt welche wiederum 8 Leveln untergeordnet sind. Wenn man alle Quizes eines Dialekts erfolgreich abgeschlossen hat - erfolgreich bedeutet hier mehr als 90\% der Antworten sind richtig - bekommt man ein Archievment welches sich in den Einstellungen betrachten lässt.\\\\
Die voliegende App wurde mit Ionic und Angular in Typescript und HTML geschrieben. Ionic ist dabei ein Framework das verschiedene Elemente wie Buttons oder Slider zur verfügung stellt und Angular erweitert HTML zum einen um viele nützliche Elemente wie Loops oder If/Else-Abfragen und setzt zum anderen eine völlig neue Abstraktionsebene auf, indem mit Komponenten gearbeitet wird welche dynamisch auf einer HTML-Seite ausgetauscht werden können. Ziel war dabei die Bavarian App mit neuer Technologie in einfacher Entwicklung sowohl für iOS als auch Android zur verfügung stellen zu können.\\\\
Im Kern funktioniert die App bereits - so wird im Hintergrund eine lokale Datenbank verwaltet die Vokabeln aus Exceltabellen beinhaltet, Daten persistiert und Assets zuordnet. Im Vordergrund wird die Datenbank abgefragt, die UI befüllt und Eingaben des Users wieder an die Datenbank zurückgegeben.\\\\Die noch zu Erfüllenden Aufgaben teilen sich auf vier große Bereiche auf:\\
Zuallererst fehlen noch einige Features wie zum Beispiel der Share-Button oder die Gratulationen am Ende eines Kurses welche im Detail dem Praktikumsbericht entnommen werden können.\\
Dann sind die Anforderungen lose der zugrundeliegenden \glqq Bavarian App\grqq{} entnommen. Wir haben uns im Grunde angeguckt was die kann und uns zuletzt ein Backlog angelegt mit Fehlenden oder unvollständigen Features welche wir sukzessive Implementiert haben. Es wäre Vorteilhaft die Anforderungen der App einmal schriftlich in einem Pflichtenheft zu erfassen und dies mit Systemtests zu überprüfen.\\
Außerdem hat die UI noch einige Schwächen. So sind die Bilder nicht immer gleich groß, allgemein sind die einzelnen Größen nicht unbedingt an Smartphones angepasst und gelegentlich wird der User über die Vorgänge in der App nur unzureichend Informiert (Mit Toasts, Alerts oder auch mit ungeschickter Menüführung). Für diesen Punkt muss die App selbstständig analysiert werden.\\
Schlußendlich kann mit ein wenig Erfahrung in Angular ein umfassendes Refactoring der Dienste (Services) der UI erfolgen. Wir haben die UI und die Datenbank strikt getrennt und versucht uns soweit möglich an S.O.L.I.D-Prinzipien zu halten - die Dienste aber haben sich zum Schluss immer mehr verknotet. Hier Abhägigkeiten zu eleminieren ist wichtig wenn die App in späteren Projektphasen noch erweitert werden soll.\\\\
Der schwierigste Teil dieses Praktikums wird das Einarbeiten zum einen in Ionic mit Angular sowie Angular selbst und zum anderen in unser Projekt sein. Da bereits alle Mechanismen bereit stehen ist eine Weiterentwicklung der Datenbank-Mechanismen vermutlich nciht unbedingt notwendig. Gegebenenfalls kann der Interactor als Schnittstelle zwischen den Datenbankmechanismen und dem Front-End erweitert werden. Das schön dabei ist, dass alles unangetastete dann mit dem alten Interface einfach weiterfunktionieren kann. Da der Fokus in diesem Projekt jetzt auf Qualitätssicherung liegt wird der Großteil der Aufgaben \glqq theoretischer\grqq{} Natur sein - soll heißen viel Analyse der zugrundeliegenden Programmen und weniger Implementieraufwand.