% !TEX root = latex_main.tex 

\section{Entwicklungsprozess}
Der Entwicklungsprozess selbst wurde zunächst keiner komplexen Systematik unterworfen. Es gab die Trennung, dass Patrick Frömel die Datenbankmechaniken und Interfaces inkl. Interactor entwickelt während Tim Romonath sich in Ionic und Angular einarbeitet und das User-Interface entwickelt.\\\\Die ersten 5 Wochen haben wir an den Wochentagen in den Semesterferien uns morgens getroffen, den aktuellen Stand besprochen und dann 6 - 8 Stunden entwickelt. Anfangs haben wir mit Agantty (Webtool zur Aufgabenplanung) gearbeitet, aber schnell festgestellt dass dies nicht unseren Bedürfnissen entsprach. Danach folgte eine Pause von 4 Wochen in welcher wir anderen Dingen nachgehen mussten. Nach dieser Pause haben wir nur schwer wieder den Einstieg in das Projekt gefunden. Zur Unterstützung haben fortan mit Trello gearbeitet, ein einfaches Webtool für Kanban-Boards.

\subsection{Agiles vorgehen}
Unser vorgehen kann am ehesten noch als \glqq agil\grqq{} beschrieben werden. Wir haben die zugrundeliegende App Stück für Stück nachgebaut - zunächst einen Funktionierenden Kern (Datenbankmechaniken, auslesen und übertragen dieser in die UI - Soundwiedergabe, grundlegendes Routing usw.), dann wichtige Kernfeatures (Bilder, Fortschritt, Archievements, usw.) und nach und nach alle anderen implementierten Features (Dialektauswahl, App zurücksetzen, usw.). \\\\Allerdings ist unser Vorgehen nicht besonders inkrementell und iterativ gewesen - eher modular chaotisch. Wir haben im Grunde gemacht was gerade anfiel und für den Großteil des Projekts auf Projektmanagement verzichtet. Das hat ob der geringen Komplexität und einer verhältnismäßig guten Codequalität funktioniert - ist aber für zukünftige Projekte nicht ratsam. Ein wirkliches, agiles Vorgehen kam erst am Ende mit Trello auf - dort wurden Aufgaben in einem Backlog gesammelt und nach und nach fertig gestellt. Allerdings gab es zu diesem Zeitpunkt nur noch ein Zwischenziel: Fertig werden.

\subsection{Verwendete Tools}
\subsubsection*{Agantty}
Agantty ist ein Webtool um Projektmanagement zu unterstützen. Es implementiert dabei ein Gantt-Diagramm, welches zeitliche Aktivitäten als Balken darstellt. Wir haben Agantty sehr schnell verworfen, da unser prinzipiell agiles Vorgehen nicht gut mit dieser Art Diagram - welches besser zu fest geplanten Projekten mit klaren Zeiten und Zwischenzielen passt (vgl. Wasserfallmodell).

\subsubsection*{Gittea Kanban-Boards}
Vor der Etablierung von Trello haben wir uns erkundigt wie die Kanban-Boards von Gittea genutzt werden können. Diese sind im Grunde eine grafische Organisationsmöglichkeit von Issues was durchaus viele Vorteile mit sich bringt - wie beispielsweise alle bekannten Gittea-Tools zum Monitoring oder die Aufgabenverteilung direkt im Repository - aber auch einen großen Nachteil weswegen wir uns dagegen entschieden haben: Für jedes Issue bekomme Beobachter des Repositorys eine Email. Beim Sammeln der fehlenden Features wäre eine regelrechte Email-Flut auf die betreuenden Personen losgebrochen.

\subsubsection*{Trello}
Wir haben zum Ende des Projekts mit Trello gearbeitet - eine einfache Implementation eines Kanban-Boards. Zu dem Zeitpunkt der Einführung von Trello existierte bereits ein grober Kern der App in Form von Datenbankmechaniken, dem grundlegenden User-Interface, ordentlichen Interfaces für Datenklassen und einer Schnittstelle zwischen Datenbank und User-Interface. Das Vorgehen Fehlende oder unvollständige Features zusammen mit Dokumentationsaufgaben in einem Backlog zu sammeln welches auch weiterhin mit Bugs und anderen Auffälligkeiten gefüllt wurde hat sehr zur Übersichtlichkeit des Projekts beigetragen. Wir haben mit 3 Bereichen gearbeitet: Dem Backlog - einem Pool an noch zu erfüllenden Aufgaben, einem Bereich für Aufgabe an denen gerade gearbeitet wird und einem für Aufgabe wo die Weiterarbeit gerade nicht möglich ist. 

\subsection{Verlauf}
Die erste, \glqq modular chaotische\grqq{} Phase war durch hohe Arbeitsbereitschaft geprägt. Durch ständiges und langes arbeiten am Projekt waren wir stets auf der Höhe des Projekts und hatten damit keine größeren Probleme mit fehlender Organisation. Wir haben uns direkt zu Beginn auf bestmögliches Einhalten der S.O.L.I.D.-Prinzipien geeinigt und daher viel Wert auf sinnvolle Interfaces gelegt. Wir haben zwar versucht mit Agantty eine Art Projektmanagement zu betreiben, dies ist aber am falschen Tool gescheitert.\\\\
Die zweite, \glqq optimistische\grqq{} Phase trat nach einer arbeitsbedingten, mehrwöchigen Pause ein. Wir hatten den Überblick über das Projekt verloren könnten diesen aber mit Trello zurückgewinnen. Leider waren wir beide auch in dieser Zeit im Broterwerb eingespannt wodurch das Projekt nun zwar organisiert, aber langsam voran ging.\\\\
Die dritte, \glqq crunchige\grqq{} Phase war von der eiligen und unsauberen (im Sinne von \glqq schlecht wartbarer\grqq{}) Implementierung letzter Features und dem fixen einiger Bugs geprägt. Nebenbei musste noch dieser Bericht geschrieben werden und es stellte sich eine gewisse Müdigkeit ein.

\subsection{Fazit}
Es wäre sehr viel Sinnvoller gewesen sich gleich zu Beginn des Praktikums mit Projektmanagementmethoden zu beschäftigen um Vorzüge etablierter Herangehensweisen zu nutzen. Mit Trello zu beginn des Projekts und von SCRUM adaptierten Sprints wäre das Ziel ein wenig klarer gewesen.\\\\Die lange Pause zwischen der ersten und zweiten Projektphase war ein großer Fehler - wenn auch kein vermeidbarer. Es fiel uns sehr schwer wieder in dem Projekt Fuss zu fassen. Dabei garnicht so sehr vom Fachlichen her sondern eher rein von der Motivation. Das Projekt wirkte bereits so fern und so erledigt was ein großer Trugschluss war. Wir sind überzeugt mit 6 oder 7 Wochen am Stück mehr erreicht haben zu können als wir schlussendlich erreicht haben.

