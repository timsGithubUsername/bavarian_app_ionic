% !TEX root = latex_main.tex 

\section{Entwicklungsprozess}
Der Entwicklungsprozess selbst wurde zunächst keiner komplexen Systematik unterworfen. Es gab die Trennung, dass Patrick Frömel die Datenbankmechaniken und Interfaces inkl. Interactor entwickelt während Tim Romonath sich in Ionic und Angular einarbeitet und das User-Interface entwickelt.\\\\Die ersten 5 Wochen haben wir an den Wochentagen in den Semesterferien uns morgens getroffen, den aktuellen Stand besprochen und dann 6 - 8 Stunden entwickelt. Anfangs haben wir mit Agantty (Webtool zur Aufgabenplanung) gearbeitet, aber schnell festgestellt dass dies nicht unseren Bedürfnissen entsprach. Danach folgte eine Pause von 4 Wochen in welcher wir anderen Dingen nachgehen mussten. Nach dieser Pause haben wir nur schwer wieder den Einstieg in das Projekt gefunden. Zur Unterstützung haben fortan mit Trello gearbeitet, ein einfaches Webtool für Kanban-Boards.

\subsection{Agiles vorgehen}
Unser vorgehen kann am ehesten noch als \glqq agil\grqq{} beschrieben werden. Wir haben die zugrundeliegende App Stück für Stück nachgebaut - zunächst einen Funktionierenden Kern (Datenbankmechaniken, auslesen und übertragen dieser in die UI - Soundwiedergabe, grundlegendes Routing usw.), dann wichtige Kernfeatures (Bilder, Fortschritt, Archievements, usw.) und nach und nach alle anderen implementierten Features (Dialektauswahl, App zurücksetzen, usw.). \\\\Allerdings ist unser Vorgehen nicht besonders inkrementell und iterativ gewesen - eher modular chaotisch. Wir haben im Grunde gemacht was gerade anfiel und für den Großteil des Projekts auf Projektmanagement verzichtet. Das hat ob der geringen Komplexität und einer verhältnismäßig guten Codequalität funktioniert - ist aber für zukünftige Projekte nicht ratsam. Ein wirkliches, agiles Vorgehen kam erst am Ende mit Trello auf - dort wurden Aufgaben in einem Backlog gesammelt und nach und nach fertig gestellt. Allerdings gab es zu diesem Zeitpunkt nur noch ein Zwischenziel: Fertig werden.

\subsection{Verwendete Tools}
\subsubsection*{Agantty}
Agantty ist ein Webtool um Projektmanagement zu unterstützen. Es implementiert dabei ein Gantt-Diagramm, welches zeitliche Aktivitäten als Balken darstellt. Wir haben Agantty sehr schnell verworfen, da unser prinzipiell agiles Vorgehen nicht gut mit dieser Art Diagram - welches besser zu fest geplanten Projekten mit klaren Zeiten und Zwischenzielen passt.

\subsubsection*{Trello}
Wir haben zum Ende des Projekts mit Trello gearbeitet - eine einfache Implementation eines Kanban-Boards. 