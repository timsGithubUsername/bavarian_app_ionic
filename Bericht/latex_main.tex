\documentclass[a4paper]{report}

\usepackage[ngerman]{babel}			%Deutsche Wortrennung
\usepackage[utf8]{inputenc}			%Umlaute! (ä, ö, ü)
\usepackage[T1]{fontenc}		    %Technisch - Umlaute werden nicht zusammengesetzt sondern als echte unicode-Zeichen verwendet

\usepackage{amsmath}				%Mathematische Befehle
\usepackage{amsthm}					%logik
\usepackage{amssymb}				%mathesymbole
\usepackage{graphicx}				%Grafikeinbindung

\usepackage{float}					%Da wir Grafiken mit figure einbinden, diese aber vermutlich nicht am Ende unseres Dokuments wiederfinden möchten. Mit diesem Paket können wir das große [H] benutzen welches dem Bild untersagt eine optimale Position zu finden

\usepackage{fancyhdr}
\pagestyle{fancy}				%Weisen LaTeX an, Kopf und Fußzeilen im Dokumentstyle zu nutzen

\setlength{\parindent}{0em}

%Im folgendem definieren wir unsere eigenen Kopf- und Fußzeilen

\lhead{Portierung der Bavarian App in Ionic}							%Kopfzeile innen
\rhead{\begin{minipage}[b]{5cm}\nouppercase{\rightmark}\end{minipage}}

\renewcommand{\thesection}{\arabic{section}}

\begin{document}
	\begin{titlepage}
		\centering
		{\scshape\Large Fortgeschrittenenpraktikum\par}
		\vspace{1.5cm}
		{\huge\bfseries Portierung der Bavarian App in Ionic\par}
		\vspace{2cm}
		{\Large\itshape Patrick Frömel, Tim Romonath\par}
		\vspace{2cm}
		{\itshape Unter Anleitung von Prof. Christoph Bockisch und PD. Dr. Hanna Fischer \par}
		\vfill
		{\large \today\par}
	\end{titlepage}
   	
   	\tableofcontents
   	
    \newpage
	% !TEX root = latex_main.tex 

Die App \glqq Welcome to Bavaria\grqq{} in der Version 3.0 ist eine Portierung der gleichnamigen Version 2.0 auf das Ionic-Framework um diese auf verschiedenen Betriebssystemen - insbesondere iOS und Android - mit geringen Entwicklungsaufwand zur Verfügung zu stellen. Die App wurde dabei im Zeitraum von Juli bis Oktober 2021 Entwickelt.\\\\
Die App beruht auf dem gedrucktem Heft \glqq Migraboarisch - Von uns fia eich. Ein Wegweiser durch den Dialekt\grqq{} welcher im Schuljahr 2016/17 von Kelheimer Schülern Entwickelt wurde. Ziel war es dabei Neuankömmlingen in Bayern den Bayrischen Dialekt näher zu bringen.\\\\
Die App wurde in der Version 1.0 von Marburger Studenten 2019/20 entwickelt und um verschiedene Sprachaufnahmen ergänzt. Für die Version 2.0 wurden weitere Sprachaufnahmen und ein primitives Archivement-System eingebracht.

\subsection{Zur Portierung}
Es wurde sich für Ionic mit dem Angular-Framework entschieden um die Vorteile der Webentwicklung bezüglich Ihrer Anpassungsfähigkeit in verhältnismäßig einfachen Frameworks zu nutzen und damit die Reichweite der App zu erweitern. Sogenannte \glqq Web-Apps\grqq{} sind weit verbreitet und stellen - vereinfacht ausgedrückt - Websites in speziellen Browsern dar so dass der Quellcode auf unterschiedlichen Betriebssystemen stets der selbe bleiben kann. So müssen bei der Wartung nicht verschiedene Programme parallel Entwickelt werden.\\\\
Ionic mit Angular wird in dieser Kombination von quasi allen Seiten propagiert und wurde in unserer unbedarftheit dankend angenommen. Ionic stellt viele mehr oder minder komplexe Bausteine zur Webentwicklung bereit (von Textfeldern über Knöpfen bis hin zu Slidern mit Animation) während Angular zum einen HTML um viele nützliche \glqq Funktionen\grqq{} erweitert (Loops, If-Else,\dots) und zum anderen komplexe Mechanismen im Hintergrund verwaltet so dass beispielsweise eine Website mit verschiedenen wechselnden Komponenten hoch dynamisch gestaltet werden kann.\\\\
In der Portierung wurde versucht sich an S.O.L.I.D-Prinzipien zu halten, auch wenn dies mit Type-Script als Programmiersprache bisweilen kompliziert gestaltete. Während die Datenbank-Mechanismen noch verhältnismäßig gut ohne Abhängigkeiten funktionieren und dank Schnittstellen (Interfaces) auch einfach ausgetauscht werden können sind die Seiten der UI sehr eng an verschiedene \glqq Services\grqq{} gebunden welche wiederum untereinander stark abhängig sind. Dies ist aber keine Notwendigkeit und das refactoren dieser kann Gegenstand eines Nachfolgeprojekts sein. Kern der Bemühungen ist ein \glqq Request-Response\grqq -Pattern in welchen ein Request über ein Interactor gestellt wird welcher wiederum von den Datenbankmechanismen eine Response-Methode aufrufen. Sehr großer Vorteil ist hier das sich die asynchrone Programmierung in Grenzen hält - eine Aufgabe kann und wird auch erst dann zuende gebracht wenn eine Antwort vorhanden ist.\\\\
\textbf{TODO DATENBANKTECHNOLOGIE}
Der Interactor
\subsection{Das Team}
\subsubsection*{Leitung}
Hanna Fischer
\subsubsection*{Koordination und Betreuung}
Hanna Fischer, Christoph Bockisch, Peter Kaspar 
\subsubsection*{Systementwicklung}
Version 3.0: Patrick Frömel, Tim Romonath\\
Version 2.0: Lea Fischbach\\
Version 1.0: Benedikt Batton, Lester Brühler, Lea Fischbach, Hannah Greß, Anh-Thy Truong, Qi Yu 
\subsubsection*{Mitarbeit}
Milena Gropp
\subsubsection*{Sprachaufnahmen}
Melanie Bajrami – Regensburg (Nordmittelbairisch)\\
Christina Böhmländer – Zwiesel (Nordmittelbairisch)\\
Felicitas Erhardt – Walleshausen (Schwäbisch-Mittelbairisches Übergangsgebiet)\\
Christian Ferstl – Regensburg (Nordmittelbairisch)\\
Edith Funk – Krumbach (Schwäbisch)\\
Markus Kunzmann – Mühldorf am Inn (Mittelbairisch)\\
Barbara Neuber – Altenstadt an der Waldnaab (Nordbairisch)\\
Michael Schnabel – Bayreuth (Ostfränkisch)\\
\subsubsection*{Übersetzung}
Arabisch: Clara Mikhail\\
Englisch: Jeffrey Pheiff\\
Tschechisch: Andrea Königsmarková
\subsubsection*{Grafiken}
Udo Butler
\subsubsection*{Beteiligte Lehrerinnen und Schülerinnen}
Peter Kaspar, Daniela von Schultz\\
Karam Alayoubi, Sali AlBoulad, Yazan Aljleilati, Ahmad AlShalabi, Mustafa AlShalabi, Arreeya Banlo, Anita Blaha, Nicolas Bruckmaier, Florian Fahle, Nico Fichte, Lukas Fischer, Tina Föhre, Nico Horn, Aisa Jusic, Alexej Kobzev, Karina Kluger, Max Kohlmeier, Samuel Konschelle, Damian Kowalski, Andreas Lening, Laura Lohwasser, Markus Müller, Felix Neuhauser, Franziska Nigl, David Nguyen, Samuel Noy, Daniel Rißmann-Toledo, Lena Ritzinger, Anna Roidl, Joshua-David Roland, Heba Sandafi, Julian Schanderl, Andreas Stemmer, Wahyu Susilo, Liviu Vasiu, Mihai Vasiu, Maria Waldhier, Markus Weber, Stefanie Weber, Amadeus-Daniel Wiselka, Burak Yılmaz, Andreas Zellner, Tobias Zinkl, Thomas Zirmer 	
	
	\newpage
	% !TEX root = latex_main.tex 

\section{Entwicklungsprozess}
Der Entwicklungsprozess selbst wurde zunächst keiner komplexen Systematik unterworfen. Es gab die Trennung, dass Patrick Frömel die Datenbankmechaniken und Interfaces inkl. Interactor entwickelt während Tim Romonath sich in Ionic und Angular einarbeitet und das User-Interface entwickelt.\\\\Die ersten 5 Wochen haben wir an den Wochentagen in den Semesterferien uns morgens getroffen, den aktuellen Stand besprochen und dann 6 - 8 Stunden entwickelt. Anfangs haben wir mit Agantty (Webtool zur Aufgabenplanung) gearbeitet, aber schnell festgestellt dass dies nicht unseren Bedürfnissen entsprach. Danach folgte eine Pause von 4 Wochen in welcher wir anderen Dingen nachgehen mussten. Nach dieser Pause haben wir nur schwer wieder den Einstieg in das Projekt gefunden. Zur Unterstützung haben fortan mit Trello gearbeitet, ein einfaches Webtool für Kanban-Boards.

\subsection{Agiles vorgehen}
Unser vorgehen kann am ehesten noch als \glqq agil\grqq{} beschrieben werden. Wir haben die zugrundeliegende App Stück für Stück nachgebaut - zunächst einen Funktionierenden Kern (Datenbankmechaniken, auslesen und übertragen dieser in die UI - Soundwiedergabe, grundlegendes Routing usw.), dann wichtige Kernfeatures (Bilder, Fortschritt, Archievements, usw.) und nach und nach alle anderen implementierten Features (Dialektauswahl, App zurücksetzen, usw.). \\\\Allerdings ist unser Vorgehen nicht besonders inkrementell und iterativ gewesen - eher modular chaotisch. Wir haben im Grunde gemacht was gerade anfiel und für den Großteil des Projekts auf Projektmanagement verzichtet. Das hat ob der geringen Komplexität und einer verhältnismäßig guten Codequalität funktioniert - ist aber für zukünftige Projekte nicht ratsam. Ein wirkliches, agiles Vorgehen kam erst am Ende mit Trello auf - dort wurden Aufgaben in einem Backlog gesammelt und nach und nach fertig gestellt. Allerdings gab es zu diesem Zeitpunkt nur noch ein Zwischenziel: Fertig werden.

\subsection{Verwendete Tools}
\subsubsection*{Agantty}
Agantty ist ein Webtool um Projektmanagement zu unterstützen. Es implementiert dabei ein Gantt-Diagramm, welches zeitliche Aktivitäten als Balken darstellt. Wir haben Agantty sehr schnell verworfen, da unser prinzipiell agiles Vorgehen nicht gut mit dieser Art Diagram - welches besser zu fest geplanten Projekten mit klaren Zeiten und Zwischenzielen passt.

\subsubsection*{Trello}
Wir haben zum Ende des Projekts mit Trello gearbeitet - eine einfache Implementation eines Kanban-Boards. 	
		
	\newpage
	% !TEX root = latex_main.tex

\section{Verwendete Technologien}
In diesem Projekt kamen einige unterschiedlichen Frameworks zum Einsatz, welche als npm-Pakete in Node.js installiert wurden. Als IDE haben wir JetBrains WebStorm genutzt, eine zur Webentwicklung modifizierten Version von Intellij. Das Management der Daten erfolgte in IndexedDB und LocalStorage.

\subsection{Node.js}
Node.js ist eine JavaScript-Laufzeitumgebung um JavaScript-Code außerhalb von Webbrowsern ausführen zu können. Durch die ereignisgesteuerte Struktur von JavaScript erwachsen einige Performancevorteile beispielsweise auf Webservern. Node.js benutzt den Paketmanager npm (Node Package Manager) um weitere Module zu installieren.

\subsection{Wichtige Node.js Module}
\subsubsection*{Ionic}
Ionic ist ein Framework um Hybride Apps als Progressive Web App zu entwickeln. Es konzentriert sich auf das Front-End, welches wir hier im Bericht als User-Interface bezeichnen. Ionic stellt verschiedene interaktive Elemente wie beispielsweise Buttons, Slider, Grids usw. zur Verfügung.

\subsubsection*{Angular}
Angular ist ein Framework um die Entwicklung des Front-Ends dynamischer, TypeScript basierter Webb Apps zu unterstützen. Es wird der HTML-Code mit direktiven erweitert um sich wiederholende Aufgaben optimiert zu implementieren (Schleifen, If-Else).

\subsubsection*{Cordova}
Cordova ist ein Framework welches aus JavaScript und HTML-Code äquivalenten, gerätespezifischen Code generiert (in unserem Fall Java und XCode).

\subsection{Datenbanken}
\subsubsection*{IndexedDB}
Die Datenbank die wir verwendet haben um die Daten der Exceltabellen zu strukturieren heißt IndexedDB. Diese ist eine NoSQL-Datenbank, welche im Grunde eine Liste von javascript-Objekten indexieren kann. Wir haben hier aber im Falle der Vokabeln nicht alle Daten hineingeschrieben, sondern nur die momentan ausgewählte Sprache und Dialekt, damit die Datenbank nicht erweitert werden muss, wenn neue Sprachen und Dialekte hinzukommen.

\subsubsection*{LocalStorage}
Um alle Benutzerdaten, wie Einstellungen und Fortschritt zu persistieren haben wir uns entschieden LocalStorage zu benutzen. Diese Datenbank, ist wesentlich simpler und einfacher zu benutzen als IndexedDB, allerdings kommt sie mit einigen Einschränkungen. Da aber unser zu Speichernden daten sowieso nicht kompliziert sind reicht diese vollkommen aus. Ein der größten Vorteile von LocalStorage ist, dass der Zugriff im Gegensatz zu IndexedDB, nicht in neuen Threads stattfindet und somit der Code leicht verständlich bleibt
		
		
	\newpage
	% !TEX root = latex_main.tex 

\section{Architektur}

\subsection{User-Interface}

\subsection{Services}

\subsection{Interfaces der Datenobjekte}
bisschen kram zu den Category, VocabularyWord usw. Interfaces

\subsection{Datenbank und Excel}
wie werden die exceltabellen asugelesen und in die datenbank gebracht

\subsection{S.O.L.I.D und Interactor-Requester}
dein sehnlichst erwarteter Architekturabschnitt
	
	\newpage
	% !TEX root = latex_main.tex

\section{Bekannte Probleme und Anregungen}
	
	\newpage
	% !TEX root = latex_main.tex

\section{Anhang}
\subsection{Script der Vorstellung}
Die Bavarian App ist ein Lernprogramm um verschiedene Bayrische Dialekte ausgehend von unterschiedlichen Sprachen wie Deutsch, Arabisch oder Tschechisch zu lernen. Es gibt zwei Modi, einen um den ausgewählten Dialekt zu lernen und einen um sich diesem Dialekt in einem Quiz zu stellen. Ein Dialekt ist in verschiedene Kategorien aufgeteilt welche wiederum 8 Leveln untergeordnet sind. Wenn man alle Quizes eines Dialekts erfolgreich abgeschlossen hat - erfolgreich bedeutet hier mehr als 90\% der Antworten sind richtig - bekommt man ein Archievment welches sich in den Einstellungen betrachten lässt.\\\\
Die voliegende App wurde mit Ionic und Angular in Typescript und HTML geschrieben. Ionic ist dabei ein Framework das verschiedene Elemente wie Buttons oder Slider zur verfügung stellt und Angular erweitert HTML zum einen um viele nützliche Elemente wie Loops oder If/Else-Abfragen und setzt zum anderen eine völlig neue Abstraktionsebene auf, indem mit Komponenten gearbeitet wird welche dynamisch auf einer HTML-Seite ausgetauscht werden können. Ziel war dabei die Bavarian App mit neuer Technologie in einfacher Entwicklung sowohl für iOS als auch Android zur verfügung stellen zu können.\\\\
Im Kern funktioniert die App bereits - so wird im Hintergrund eine lokale Datenbank verwaltet die Vokabeln aus Exceltabellen beinhaltet, Daten persistiert und Assets zuordnet. Im Vordergrund wird die Datenbank abgefragt, die UI befüllt und Eingaben des Users wieder an die Datenbank zurückgegeben.\\\\Die noch zu Erfüllenden Aufgaben teilen sich auf vier große Bereiche auf:\\
Zuallererst fehlen noch einige Features wie zum Beispiel der Share-Button oder die Gratulationen am Ende eines Kurses welche im Detail dem Praktikumsbericht entnommen werden können.\\
Dann sind die Anforderungen lose der zugrundeliegenden \glqq Bavarian App\grqq{} entnommen. Wir haben uns im Grunde angeguckt was die kann und uns zuletzt ein Backlog angelegt mit Fehlenden oder unvollständigen Features welche wir sukzessive Implementiert haben. Es wäre Vorteilhaft die Anforderungen der App einmal schriftlich in einem Pflichtenheft zu erfassen und dies mit Systemtests zu überprüfen.\\
Außerdem hat die UI noch einige Schwächen. So sind die Bilder nicht immer gleich groß, allgemein sind die einzelnen Größen nicht unbedingt an Smartphones angepasst und gelegentlich wird der User über die Vorgänge in der App nur unzureichend Informiert (Mit Toasts, Alerts oder auch mit ungeschickter Menüführung). Für diesen Punkt muss die App selbstständig analysiert werden.\\
Schlußendlich kann mit ein wenig Erfahrung in Angular ein umfassendes Refactoring der Dienste (Services) der UI erfolgen. Wir haben die UI und die Datenbank strikt getrennt und versucht uns soweit möglich an S.O.L.I.D-Prinzipien zu halten - die Dienste aber haben sich zum Schluss immer mehr verknotet. Hier Abhägigkeiten zu eleminieren ist wichtig wenn die App in späteren Projektphasen noch erweitert werden soll.\\\\
Der schwierigste Teil dieses Praktikums wird das Einarbeiten zum einen in Ionic mit Angular sowie Angular selbst und zum anderen in unser Projekt sein. Da bereits alle Mechanismen bereit stehen ist eine Weiterentwicklung der Datenbank-Mechanismen vermutlich nciht unbedingt notwendig. Gegebenenfalls kann der Interactor als Schnittstelle zwischen den Datenbankmechanismen und dem Front-End erweitert werden. Das schön dabei ist, dass alles unangetastete dann mit dem alten Interface einfach weiterfunktionieren kann. Da der Fokus in diesem Projekt jetzt auf Qualitätssicherung liegt wird der Großteil der Aufgaben \glqq theoretischer\grqq{} Natur sein - soll heißen viel Analyse der zugrundeliegenden Programmen und weniger Implementieraufwand.
\end{document}